\section{Potential vulnerabilities}
\label{sec:potential_vulnerabilities}

\subsection{Contract-side attacks}
\subsubsection{Re-entrancy attacks}
Withdrawal is implemented such that re-entrancy attacks do not apply. In
particular, we can benefit from having separate contracts for different games.
As each game has its own contract, we do not need to save the amount of Ether
wagered by players in any way. Instead, when withdrawing the wagered funds, the
full balance of the game contract is transferred over to the winner. Moreover,
the phase of the game is updated before the \mintinline{solidity}{transfer}
call, ensuring that any re-entrance attempt is thwarted.

\subsubsection{AFK abusing}
As discussed in \cref{sec:implementation-contract-leaver-reporting}, leaver
reporting and checking can be easily implemented incorrectly. While the idea
that drove the current implementation should be correct, the implementation may
potentially be flawed. Such flaws may be hard to spot, and their exploitation
would have a very high impact, that is, basically allowing to win every game.

For this reason, throughout testing would be needed to make sure that AFK
reporting and AFK checking is correct.

\subsubsection{Denial of Service}
Denial of service attacks are thwarted by the possibility to report the
opponent player for leaving. However, reporting a player still has a gas cost
(even though it is low), and the malicious player can still delay the
continuation of the game by four blocks per action (about $40 - 80$ seconds per
move).

\subsubsection{Cheating detection}
Cheating detection, with the scheme described in
\cref{sec:implementation-contract-board}, is possible only when the winner
submits its board for verification. Moreover, it is also possible that the
opponent has also cheated. The current implementation assumes the loyalty of
the loser, in the case in which the winner cheated. A different implementation
could be used to:
\begin{itemize}
	\item Detect cheating in ``real-time'', that is, when the cell is checked (not
	      possible using Merkle proofs, as we can only verify belonging to a certain
	      Merkle root hash)
	\item Decide the action to take in the event that both players have cheated
\end{itemize}

\subsection{Client-side attacks}
\subsubsection{Weak salts}
The developed client uses cryptographically secure random salts of 256 bits.
However, anybody could write an alternative client. If the client (maliciously)
uses weak salts when creating the Merkle tree, it could lead to some attacks.

In particular, if using fixed salt or no salt at all, it could be theoretically
possible to compute the opponent's board. As the board's size is $8 \times 8$,
a brute force of $64$ bits would be required, which is not considered safe for
modern standards ($80$ bits is usually considered as the minimum number of bits
required to prevent brute force attacks). Moreover, the total number of
possible different configurations is also much lower than $2^{64}$. The attack
would follow these steps (assuming no salt or known fixed salt):
\begin{enumerate}
	\item Get from the blockchain the Merkle root of the opponent
	\item For each configuration of $64$ bits, $10$ of which at one and $64-10$ at zero:
	      \begin{enumerate}
		      \item Compute the Merkle tree with the guessed bit configuration
		      \item Compare the root of the obtained Merkle tree with the opponent's one
		      \item If it matches, we have found the opponent's board configuration
	      \end{enumerate}
\end{enumerate}
Notice that, if using no salt, a fixed one, or very weak ones, precomputed
rainbow tables could be used to speed up Merkle tree construction noticeably.

\subsubsection{Cross-site scripting}
As mentioned in \cref{sec:implementation-front-end}, XSS vulnerabilities are
mostly mitigated by the use of React. While XSS is possible in React, most
attack vectors are sanitized by React. Unless writing user-controlled data in
specific places that are well-known for being injectable (for example
\mintinline{HTML}{<a href="javascript:alert(1)"></a>}, or using React's
\mintinline{javascript}{dangerouslySetInnerHTML} attribute), React is safe.

