\section{Implementation choices}

\subsection{Contracts}
The first decision was to split the required functionality in two contracts:
the first contract only handles creation and join of games (i.e. deployment and
setup of a game contract), whereas the second contract is used to actually play
the game. This decision was taken for two main reasons:
\begin{itemize}
	\item Code readability and auditability: while it would be possible to implement
	      everything in a single contract, code readability and auditability would suffer
	      from such a choice. Splitting the functionality in multiple contracts allows
	      reasoning about them separately.
	\item Security: as the contract would have to deal with multiple games at once, it
	      would be more likely to introduce hard-to-spot vulnerabilities. For example, it
	      would be possible for the contract to erroneously act on the wrong game. By
	      having each game be played on a different contract, this scenario is avoided.
\end{itemize}
Of course, this also implies a higher cost of a complete game as deploying a
contract is costly in terms of the required Gas.

\subsubsection{Joining a random game}
The contract keeps a list of games that can be joined. When attempting to join
a random game, a starting index is computed pseudo-randomly. In particular, the
starting index is calculated as following:
\begin{itemize}
	\item First, the keccak256 hash with the hash of the previous block, the address of
	      the caller, and a nonce (which is incremented after each use) is computed. This
	      should guarantee that a different hash is always computed, even for
	      transactions of the same block sent by the same address.
	\item Then, a modulo operation with the length of the list of games is used.
\end{itemize}
This starting index is then used to traverse the list of joinable games.
Iterating on the list is needed to search for a game that was not created by
the caller. In most cases, however, the loop body will only be executed once.

\subsubsection{Board size, encoding and fleet}
\label{sec:implementation-contract-board}
The chosen board size is a fixed $8 \times 8$ board with a total of $10$ ships.

Following the project requirements, a Merkle tree for proving belonging is
computed starting from the board's configuration. Each cell may contain a ship
(encoded as boolean \mintinline{solidity}{true}) or not
(\mintinline{solidity}{false}). The content of the cell is hashed together with
a cryptographically strong random salt. Additionally, also the index in the
linearized matrix is hashed. Only the root of the Merkle tree is sent to the
contract, which uses it to validate proofs sent by the client.

This encoding, however, does not allow to easily check some conditions on the
fleet. In particular, given a board and a set of rules about the fleet
composition, the procedure to determine if the board respects the rules is
quite complex, as it requires checking that there exist an interpretation of
the board that satisfies the required properties (a fast-growing tree of
recursive calls is likely required). The choice in this case was to relax board
checking: as no requirements were set regarding the fleet composition, the
decision was to allow any fleet composition, with the only requirement that
every ship must be withing the board's boundaries, and that the total number of
``ship cells'' is the fixed one. Each player can basically choose its own
strategy.

A more efficient encoding using triplets of $\langle coords, size,
	direction\rangle$ could be used to simplify board checking. Given a set of
triplets, it is trivial to verify that they respect a certain fleet
composition, as well as other properties (e.g. no collisions between ships).
However, it is not as trivial to come up with a scheme that allows to verify
whether a shot hits or misses the opponent's ship without spoiling other
fundamental properties, such as hiding.

\subsubsection{Leaver reporting and checking}
\label{sec:implementation-contract-leaver-reporting}
Implementing correctly leaver (AFK) reporting and checking proved to be more
complex than what it may seem at first glance.

When implementing this feature, it is important to make sure that this feature
cannot be abused to win games. For instance, it may happen that a player is
accused on being AFK, but is also unable to take any action (defaulting to a
loss). Note that giving players the ability to declare they are not AFK is also
very risky (if not wrong) as it may be abused to lock the funds of the opponent
until the opponent goes AFK (e.g. goes to sleep).

The solution implemented is the following: check that the reported player can
do an action. If not, disallow reporting it for AFK. Additionally, reporting
for AFK is allowed only during some phases. This approach, as error-prone as it
may be, is likely the best one if implemented correctly.

\subsubsection{Verifying Merkle proofs}
For Merkle proofs verification, the
\href{https://docs.openzeppelin.com/contracts/4.x/api/utils\#MerkleProof}{\color{blue}OpenZeppelin
	MerkleProof} library was used. In addition to simple Merkle proofs, the library
allows for multi proof verification, which is done by re-building partially the
tree up to the root efficiently. Using multi proofs allowed saving up about
$\frac{1}{5}$ of the gas used for board checking.

\subsection{Front-end}
\label{sec:implementation-front-end}
The front-end has been implemented using \href{https://react.dev/}{\color{blue}
	React} and \href{https://reactrouter.com/en/main}{\color{blue} React-router}.
These two libraries allow to quickly design and implement a front-end offering
a good user experience.

The use of React-router also greatly simplifies the code: loaders and actions
are used to load data from the contract, whereas actions allow sending
transactions. Overall, this allows to keep data loading and mutation separated
from data visualization and user interaction.

An important choice during implementation regards the generation of salts for
the Merkle tree. As salts need to be random in order to prevent attacks (more
on this in \cref{sec:potential_vulnerabilities}), cryptographically secure
random numbers must be used. Fortunately, the
\href{https://developer.mozilla.org/en-US/docs/Web/API/Crypto}{\color{blue}Crypto
	API} of modern browsers allows generating cryptographically strong random
numbers easily. The front-end implementation uses this API in order to generate
secure salts which allow preventing some attacks.

To allow seamless reloading of the battleship game webpage, both the board and
the Merkle tree of the player are saved in the indexedDB of the browser (using
\href{https://github.com/localForage/localForage}{\color{blue}localForage}).
Possible attacks exploiting XSS vulnerabilities to steal the board
configuration do not work as React is safe against XSS by
default\footnote{Although this is not true in every case, the potential attack
	vectors for XSS vulnerabilities are very few and easy to avoid in React.}.

For the creation of Merkle trees and related Merkle proofs, the library
\href{https://github.com/OpenZeppelin/merkle-tree}{\color{blue}@openzeppelin/merkle-tree}
was used.

